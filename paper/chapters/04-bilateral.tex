% Chapter 4: The Bilateral Construction Property
% Two Generals Protocol Paper (v2)

The core theoretical contribution is the \emph{bilateral construction property}: the triple proofs $\Triple{A}$ and $\Triple{B}$ form a cryptographic knot where neither can exist without the other being constructible.

\begin{theorem}[Bilateral Constructibility]
\label{thm:bilateral}
If party $A$ can construct $\Triple{A}$, then party $B$ can construct $\Triple{B}$, and vice versa:
\[
\exists \Triple{A} \Leftrightarrow \exists \Triple{B}
\]
\end{theorem}

\begin{proof}
We prove the forward direction; the reverse is symmetric.

Suppose Alice can construct $\Triple{A} = \Sign{A}{\Double{A} \| \Double{B}}$.

\textbf{Step 1:} Alice has $\Double{B}$. By definition, $\Double{B} = \Sign{B}{\Com{B} \| \Com{A}}$, so Bob received Alice's commitment.

\textbf{Step 2:} For Alice to have $\Double{B}$, Bob must have constructed it, meaning Bob had $\Com{A}$.

\textbf{Step 3:} Since Bob has $\Com{A}$, Bob can construct $\Double{B}$. And since Alice is flooding $\Double{A}$, Bob will receive it under fair-lossy conditions.

\textbf{Step 4:} Upon receiving $\Double{A}$, Bob can construct $\Triple{B} = \Sign{B}{\Double{B} \| \Double{A}}$.

Therefore, if $\Triple{A}$ exists, $\Triple{B}$ is constructible under fair-lossy conditions.
\end{proof}

\subsection{The Cryptographic Knot}

Traditional protocols create a chain of acknowledgments where each link could be the ``last message'' that fails:
\[
\text{MSG} \rightarrow \text{ACK} \rightarrow \text{ACK-of-ACK} \rightarrow \cdots
\]

TGP creates a \emph{knot}:
\begin{center}
\begin{tikzpicture}[node distance=2cm]
    \node (TA) {$\Triple{A}$};
    \node (TB) [right=of TA, xshift=1cm] {$\Triple{B}$};
    \node (DA) [below=of TA] {$\Double{A}$};
    \node (DB) [below=of TB] {$\Double{B}$};

    \draw[<->, ultra thick, red] (TA) -- (TB) node[midway, above] {bilateral};
    \draw[->] (DA) -- (TA);
    \draw[->] (DB) -- (TA);
    \draw[->] (DA) -- (TB);
    \draw[->] (DB) -- (TB);
\end{tikzpicture}
\end{center}

$\Triple{A}$ embeds $\Double{B}$; $\Triple{B}$ embeds $\Double{A}$. Neither can exist without the other being constructible. There is no ``last message''---there is mutual cryptographic entanglement.

\subsection{Why This Breaks Gray's Attack}

Gray's impossibility relies on the ``drop last message'' attack: find an execution where one party decides ATTACK, then remove the last message to create asymmetry.

This attack fails against TGP because:

\begin{enumerate}
    \item The attack key is not triggered by receiving a message---it \emph{emerges} from state
    \item Removing $\Triple{B}$ from a schedule where Alice attacks means Alice never had $\Triple{B}$
    \item But if Alice never had $\Triple{B}$, the attack key never emerged for Alice either
    \item Therefore, Alice never attacked in the modified execution
\end{enumerate}

The bilateral construction makes the ``pivotal message'' impossible: any message whose removal would cause asymmetry cannot exist, because asymmetry requires one party to have the attack key while the other lacks it---but the knot structure prevents this.

\subsection{Formal Statement}

\begin{theorem}[No Pivotal Message]
\label{thm:no-pivotal}
For any TGP execution and any message $m$, removing $m$ cannot create an asymmetric outcome.
\end{theorem}

This is proven exhaustively in Lean 4 via \texttt{tgp\_no\_pivotal} using \texttt{native\_decide} over all 256 channel state combinations.
