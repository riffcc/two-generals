% Chapter 7: Formal Proofs
% Two Generals Protocol Paper

\begin{theorem}[Safety --- Unconditional]
\label{thm:safety}
No execution of the protocol results in asymmetric decisions. \textbf{This holds for all adversary behaviors, including total channel failure (NoChannel).}
\end{theorem}

\begin{proof}
The attack key is a tripartite construction requiring:
\begin{enumerate}
    \item Virtual artifact $V$ (requires $D_A \land D_B$)
    \item Alice's response (requires $V$ and Alice having $D_B$)
    \item Bob's response (requires $V$ and Bob having $D_A$)
\end{enumerate}

The attack key exists if and only if \emph{all three} components exist. By case analysis on any Boolean combination of $(d_a, d_b, a\_responds, b\_responds)$:
\begin{itemize}
    \item If $d_a \land d_b \land a\_responds \land b\_responds$: attack key exists $\Rightarrow$ both $\Attack$
    \item Otherwise: attack key is $\mathsf{none}$ $\Rightarrow$ both $\Abort$
\end{itemize}

There is no Boolean assignment where exactly one party can attack. The Lean 4 theorem \texttt{gray\_unreliable\_always\_symmetric} proves this by exhaustive case analysis with \emph{no fair-lossy assumption}.

\textbf{Critical:} This is pure case analysis on the attack key structure. The channel model is irrelevant to safety---even under NoChannel (total message loss), outcomes are symmetric (both abort).
\end{proof}

\begin{theorem}[Liveness]
\label{thm:liveness}
Under fair-lossy channels with delivery probability $p > 0$, the probability that both parties reach a coordinated decision approaches 1.
\end{theorem}

\begin{proof}
The protocol uses six packets: $C_A, C_B$ (commitments), $D_A, D_B$ (double proofs), $T_A, T_B$ (triple proofs). Each phase requires delivery of one message type. With continuous flooding:
\begin{itemize}
    \item Phase 1: $\Pr[\text{both receive } C] = 1$ (fair-lossy)
    \item Phase 2: $\Pr[\text{both receive } D] = 1$ (fair-lossy)
    \item Phase 3: $\Pr[\text{both receive } T] = 1$ (fair-lossy)
\end{itemize}

The probability of completing all phases is $1$ under fair-lossy conditions. The attack key emerges when both parties have received $T$ (which embeds $D$, which embeds $C$).

With finite deadline $\tau$ and per-message delivery probability $p$, the probability of completing within $\tau$ is:
\[
\Pr[\text{complete}] = 1 - (1-p)^{n}
\]
where $n$ is the number of transmission attempts. For continuous flooding at rate $r$ messages/second over duration $\tau$:
\[
\Pr[\text{complete}] = 1 - (1-p)^{r\tau}
\]

With $p = 0.01$, $r = 1000$, $\tau = 10$s: $\Pr[\text{complete}] > 1 - 10^{-1565}$.
\end{proof}

\begin{remark}[Physical Interpretation of $10^{-1565}$]
A failure probability of $10^{-1565}$ is so fantastically small that \textbf{you would need to run this protocol once per picosecond, on every atom in a trillion universes, from the Big Bang until the heat death of the cosmos, and you still would not expect to see a single failure}. For context: there are approximately $10^{80}$ atoms in the observable universe. The probability $10^{-1565}$ is $10^{1485}$ times smaller than one divided by that count. This is not a probability in any meaningful physical sense---it is a formality. The protocol \emph{cannot} fail by random chance; it can only fail through implementation defects, hardware errors, or environmental pathologies not captured by the fair-lossy model.
\end{remark}

\begin{theorem}[Validity]
\label{thm:validity}
If both parties intend to attack and the network is fair-lossy, both decide $\Attack$.
\end{theorem}

\begin{proof}
Both parties begin by flooding commitments. Under fair-lossy conditions, both eventually receive the counterparty's commitment, construct double proofs, exchange those, then construct triple proofs. When both have $T_A$ and $T_B$, the attack key emerges and both decide $\Attack$.
\end{proof}
