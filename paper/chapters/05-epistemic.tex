% Chapter 5: The Epistemic Fixpoint
% Two Generals Protocol Paper (v2)

The bilateral construction property achieves something remarkable: a finite cryptographic structure that encodes sufficient epistemic depth for coordination. This section provides formal epistemic logic treatment of why TGP resolves Gray's impossibility.

\subsection{Epistemic Logic Background}

Following Fagin et al.~\cite{fagin1995reasoning} and Halpern-Moses~\cite{halpern1990knowledge}, we use standard modal logic notation:

\begin{itemize}
    \item $\Know{X}{\phi}$: ``Party $X$ knows $\phi$''
    \item $\Know{A}{\Know{B}{\phi}}$: ``$A$ knows that $B$ knows $\phi$''
    \item $C(\phi)$: Common knowledge of $\phi$ --- the infinite conjunction:
    \[
    C(\phi) \equiv \phi \land \Know{A}{\phi} \land \Know{B}{\phi} \land \Know{A}{\Know{B}{\phi}} \land \Know{B}{\Know{A}{\phi}} \land \cdots
    \]
\end{itemize}

\subsection{Gray's Impossibility Restated}

Gray~\cite{gray1978notes} and Halpern-Moses~\cite{halpern1990knowledge} proved:

\begin{theorem}[Common Knowledge Impossibility --- Gray/Halpern-Moses]
In any system where communication is not guaranteed, common knowledge of any fact cannot be achieved through finite message sequences.
\end{theorem}

The proof relies on the observation that each epistemic level requires explicit acknowledgment:
\begin{center}
$\Know{A}{\phi} \Rightarrow$ message delivered $\Rightarrow$
$\Know{B}{\Know{A}{\phi}} \Rightarrow$ ACK delivered $\Rightarrow \cdots$
\end{center}

Any message in this chain could be ``the last'' that fails, preventing the next level from being established.

\subsection{The Paradigm Shift: Construction vs Communication}

Our resolution rests on a fundamental reframing:

\begin{center}
\fbox{
\begin{minipage}{0.85\columnwidth}
\textbf{Gray's Model:} Knowledge is \emph{transferred} via message exchange.\\[0.5em]
\textbf{Our Model:} Knowledge is \emph{embedded} in cryptographic structure.
\end{minipage}
}
\end{center}

The artifact $\Triple{A}$ does not \emph{communicate} that Alice knows Bob knows---its \emph{existence proves} that Alice has Bob's $\Double{B}$, which proves Bob had Alice's $\Com{A}$, which proves the mutual knowledge chain terminates.

\subsection{Formal Definition: Epistemic Fixpoint}

\begin{definition}[Epistemic Fixpoint]
\label{def:fixpoint}
A protocol achieves an \emph{epistemic fixpoint} if there exists an artifact where construction by one party guarantees constructibility by the counterparty:
\[
\mathsf{constructed}(\Triple{A}) \Rightarrow \mathsf{constructible}(\Triple{B})
\]
\end{definition}

\begin{theorem}[TGP Achieves Epistemic Fixpoint]
\label{thm:fixpoint}
The triple proof pair $(\Triple{A}, \Triple{B})$ satisfies Definition~\ref{def:fixpoint}:
\[
\exists \Triple{A} \Leftrightarrow \exists \Triple{B} \text{ (under fair-lossy)}
\]
\end{theorem}

\begin{proof}
Suppose $\Triple{A}$ exists. By construction:
\begin{align*}
\Triple{A} &= \Sign{A}{\Double{A} \| \Double{B}} \\
\Double{B} &= \Sign{B}{\Com{B} \| \Com{A}} \subseteq \Triple{A}
\end{align*}

Therefore Alice possesses $\Double{B}$, which proves:
\begin{enumerate}
    \item Bob constructed $\Double{B}$ (signature verification)
    \item Bob had $\Com{A}$ when constructing $\Double{B}$ (embedded in $\Double{B}$)
    \item Bob can construct $\Triple{B}$ once he receives $\Double{A}$
\end{enumerate}

Since Alice floods $\Double{A}$, and the channel is fair-lossy:
\begin{itemize}
    \item Bob will receive $\Double{A}$ with probability 1
    \item Bob can construct $\Triple{B}$
\end{itemize}

Thus: $\Triple{A} \Rightarrow \mathsf{constructible}(\Triple{B})$

By symmetric argument: $\Triple{B} \Rightarrow \mathsf{constructible}(\Triple{A})$

The mutual implication creates the fixpoint:
\[
\Triple{A} \Leftrightarrow \Triple{B} \text{ (under fair-lossy)}
\]

This is not an infinite regress---it is a \textbf{self-referential cryptographic entanglement} where each half proves the other's constructibility through its own structure.
\end{proof}

\subsection{Why Cryptography Resolves the Impossibility}

The key insight is that \textbf{cryptographic signatures create unforgeable proofs of prior possession}.

When Alice signs $\Triple{A}$ over $\Double{B}$, she produces permanent, verifiable evidence that she possessed $\Double{B}$ at signing time. This evidence is \emph{self-certifying}---no additional messages needed.

\begin{proposition}[Self-Certification]
Each proof level in TGP is self-certifying: verifying the signature on $\Triple{X}$ simultaneously proves:
\begin{enumerate}
    \item $X$ created $\Triple{X}$ (signature validity)
    \item $X$ possessed $\Double{X}$ and $\Double{Y}$ (embedded in $\Triple{X}$)
    \item $X$ possessed all four commitments (embedded in the doubles)
\end{enumerate}
\end{proposition}

This transforms the problem from ``How do I know you received my message?'' to ``What does your cryptographic artifact prove you possessed?''

\subsection{The Epistemic Depth Table}

\begin{center}
\begin{tabular}{lcll}
\toprule
\textbf{Level} & \textbf{Depth} & \textbf{Artifact} & \textbf{Epistemic Content} \\
\midrule
Commitment & 0 & $\Com{X}$ & ``I intend to attack'' \\
Double & 1 & $\Double{X}$ & $\Know{X}{\Com{Y}}$ \\
Triple & 2+ & $\Triple{X}$ & $\Know{X}{\Know{Y}{\Com{X}}}$ + bilateral \\
\bottomrule
\end{tabular}
\end{center}

The triple level achieves sufficient epistemic depth because the bilateral construction property creates a closed loop: if Alice can construct $\Triple{A}$, Bob can construct $\Triple{B}$, and vice versa. The attack key \emph{emerges} from this bilateral state.

\subsection{The Elevator, Not the Ladder}

A clarifying metaphor for the paradigm shift:

\begin{center}
\fbox{
\begin{minipage}{0.85\columnwidth}
\textbf{Gray's Model:} To reach epistemic level $n$, you must climb $n$ rungs, each requiring a separate message.\\[0.5em]
\textbf{TGP's Model:} The triple proofs form a \emph{knot} that reaches sufficient depth in three phases.
\end{minipage}
}
\end{center}

The epistemic ladder is still infinitely tall in \emph{theory}. But TGP does not need to reach the top---it only needs sufficient epistemic depth for the attack key to emerge with bilateral guarantees. Three levels suffice.

\subsection{Cryptography as Epistemic Machinery}

A potential objection: ``You've enriched Gray's model with cryptography---doesn't that invalidate the comparison?''

We reject this framing. Cryptography is \emph{not} an oracle or black box external to the model. From the perspective of distributed systems theory:

\begin{itemize}
    \item A ``signature'' is just a deterministic function: $\mathsf{sign} : (\mathsf{sk}, m) \mapsto \sigma$
    \item Verification is another function: $\mathsf{verify} : (\mathsf{pk}, m, \sigma) \mapsto \{true, false\}$
    \item No magic oracles, no shared randomness, no out-of-band coordination
\end{itemize}

TGP is \textbf{still just deterministic state machines passing finite-length bitstrings over lossy channels}---exactly the class of systems Gray's theorem was intended to cover. We have not changed the system class; we have enriched the local transition function in a way Gray's proof implicitly excluded.

If their theorem was meant to cover \emph{all} message-passing protocols on unreliable channels, including crypto-enhanced ones, TGP is a counterexample. If their theorem was intended only for protocols without structured cryptographic introspection, then ``the Coordinated Attack Problem is impossible'' was always an overstatement of the actual result.
