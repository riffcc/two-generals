% Chapter 8: The Protocol of Theseus
% Two Generals Protocol Paper

The name ``Protocol of Theseus'' is not merely branding---it captures a deep truth about the protocol's structure.

\subsection{The Philosophical Foundation}

\paragraph{The Ship of Theseus Paradox.}
If Theseus's ship has each plank replaced over time, is the resulting vessel still ``Theseus's ship''? The identity seems to depend on continuity of structure rather than identity of components.

\paragraph{The Protocol of Theseus Property.}
TGP exhibits an analogous property: \emph{if you remove any message---or indeed, any subset of messages---does the protocol still guarantee symmetric outcomes?}

\textbf{Answer: Yes.}

The protocol's correctness depends on \emph{cryptographic structure}, not on which specific message instances are delivered. Any packet carrying $\Triple{A}$ will do; the protocol doesn't care which copy arrives.

This directly refutes Gray's ``last message'' problem:
\begin{itemize}
    \item \textbf{Gray's model:} There exists a critical ``last message'' whose loss causes asymmetry
    \item \textbf{TGP:} All messages are fungible; continuous flooding ensures eventual delivery; no message is special
\end{itemize}

\subsection{Formal Statement}

\begin{proposition}[Protocol of Theseus Property]
Let $\mathcal{M}$ be the multiset of messages sent during a TGP execution. For any proper subset $\mathcal{M}' \subset \mathcal{M}$ removed by an adversary:

If the remaining messages $\mathcal{M} \setminus \mathcal{M}'$ still constitute a fair-lossy channel (i.e., at least one copy of each message type eventually delivers), then the protocol achieves symmetric outcomes.
\end{proposition}

\begin{proof}
By the bilateral construction property (Theorem~\ref{thm:bilateral}), if either party constructs their $T$, the counterparty's $T$ is constructible. The proof artifact itself guarantees this---independent of which specific message copy delivered the components. Continuous flooding ensures that as long as the channel remains fair-lossy after adversarial removal, eventual delivery occurs. The symmetry guarantee follows from the cryptographic structure, not from any particular message.
\end{proof}

\subsection{Why Gray's Proof Fails on TGP}

Gray's impossibility proof has a specific structure:

\begin{enumerate}
    \item Consider any finite protocol $P$ where both parties decide $\Attack$.
    \item Let $m$ be the \textbf{last message sent} in that execution.
    \item Construct a new execution where $m$ is lost.
    \item The sender of $m$ has the same local state, so must still decide $\Attack$.
    \item The receiver has \emph{less} information, so may decide $\Abort$.
    \item Therefore asymmetric outcomes are possible. \qed
\end{enumerate}

\textbf{This argument fails on TGP} because step (4) is false: there is no message whose removal changes one party's decision without changing the other's.

\begin{theorem}[No Critical Last Message]
\label{thm:nolast}
In TGP, for any successful $\Attack$ execution, there exists no message $m$ such that:
\begin{itemize}
    \item Removing $m$ causes one party to decide $\Abort$
    \item While the other party still decides $\Attack$
\end{itemize}
\end{theorem}

\begin{proof}[Proof Sketch]
Consider any message $m$ in a successful execution. Two cases:

\textbf{Case 1:} $m$ is not on a minimal dependency path for either party's fixpoint condition.
Then both parties still reach $\mathsf{FIXPOINT\_OK}$ via redundant proof copies. Both still $\Attack$.

\textbf{Case 2:} $m$ is on a minimal dependency path for at least one party's fixpoint.
By bilateral construction, if $m$ carries information critical for Alice's fixpoint, then $m$ (or its contents) must also be critical for Bob's. If $m$ is lost and no equivalent arrives before deadline:
\begin{itemize}
    \item Alice cannot reach $\mathsf{FIXPOINT\_OK}$ $\Rightarrow$ Alice $\Abort$s
    \item Bob cannot reach $\mathsf{FIXPOINT\_OK}$ $\Rightarrow$ Bob $\Abort$s
\end{itemize}
Both $\Abort$ symmetrically. No asymmetry.

The key insight: any message ``critical'' for $\Attack$ is \emph{symmetrically critical}---its absence causes both parties to fail the fixpoint condition, not just one.
\end{proof}

\subsection{Empirical Validation: The Packet Removal Test}

We validated Theorem~\ref{thm:nolast} empirically by systematically removing each packet from successful executions:

\begin{center}
\begin{tabular}{lc}
\toprule
\textbf{Test Configuration} & \textbf{Result} \\
\midrule
Total test runs & 10,500 \\
Packet loss rates tested & 0--98\% \\
Packets removed per run & Each, one at a time \\
Asymmetric outcomes observed & \textbf{0} \\
\bottomrule
\end{tabular}
\end{center}

For each of the 10,500 successful runs, we:
\begin{enumerate}
    \item Recorded the complete message trace
    \item Systematically removed each packet one at a time
    \item Verified the resulting outcome
\end{enumerate}

In \textbf{every case}, removing a packet resulted in either:
\begin{itemize}
    \item Both parties still reaching $\Attack$ (via redundant proof copies), or
    \item Both parties reaching $\Abort$ (symmetric failure to achieve fixpoint)
\end{itemize}

\textbf{Zero asymmetric outcomes were observed.} This empirically confirms that Gray's ``last message'' argument does not apply to TGP: there is no packet whose removal produces the asymmetry his proof requires.
