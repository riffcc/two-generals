\documentclass[11pt,a4paper]{article}

\usepackage{amsmath,amssymb,amsthm}
\usepackage{hyperref}
\usepackage{graphicx}
\usepackage{booktabs}

\newtheorem{theorem}{Theorem}
\newtheorem{lemma}[theorem]{Lemma}
\newtheorem{corollary}[theorem]{Corollary}
\newtheorem{definition}{Definition}

\title{The Third Can of Paint:\\Fair Exchange Without a Trusted Third Party}

\author{Wings\\Riff Labs\\wings@riff.cc}

\date{January 2026}

\begin{document}

\maketitle

\begin{abstract}
We present a construction that achieves fair exchange between two parties
without requiring a trusted third party (TTP). The classical result of
Pagnia and G\"artner (1999) states that fair exchange is impossible without
a TTP. We show this impossibility can be circumvented through \emph{bilateral
construction}: a shared artifact that emerges only when both parties
contribute, and cannot exist otherwise. We call this the ``third can of
paint'' construction.

Our concrete instantiation uses \emph{bilateral Schnorr signatures}: a
four-phase protocol (C $\to$ D $\to$ T $\to$ S) where the Schnorr challenge
$e = H(R, P, m, \text{attack\_key})$ is bound to an \emph{attack key} that
can only be computed after both parties complete the bilateral T-exchange.
Since neither party can compute the challenge without both T proofs, neither
can compute their partial signature first. The signature \emph{emerges} from
bilateral completion---it cannot exist otherwise.

Our construction achieves: (1) no party ``goes first''---partial signatures
are impossible until bilateral completion; (2) symmetric outcomes under all
channel conditions including total failure; (3) MuSig-style key aggregation
preventing rogue-key attacks. Reference implementation in Rust with formal
proofs in Lean 4 (zero \texttt{sorry} statements). This work extends the
Two Generals Protocol (TGP), applying the same bilateral construction
principle to fair exchange.
\end{abstract}

\section{Introduction}

\subsection{The Fair Exchange Problem}

Fair exchange is a fundamental problem in distributed systems and
cryptographic protocols. Two parties, Alice and Bob, wish to exchange
items such that either both receive what they want, or neither does.
The canonical examples include:

\begin{itemize}
    \item \textbf{Contract signing}: Neither party wants to be bound
          while the other is not
    \item \textbf{Atomic swaps}: Neither party wants to lose their
          cryptocurrency while the other keeps theirs
    \item \textbf{Certified email}: The sender wants proof of receipt;
          the receiver wants the content
\end{itemize}

\subsection{The Classical Impossibility}

Pagnia and G\"artner \cite{pagnia1999} proved that fair exchange is
impossible without a trusted third party (TTP). Their argument is
intuitive: someone must ``go first,'' and whoever goes first can be
cheated. Therefore, an external arbiter is required to hold both items
and release them atomically.

The TTP requirement is problematic:
\begin{itemize}
    \item \textbf{Single point of failure}: If the TTP fails, no exchange
    \item \textbf{Single point of trust}: All parties must trust the TTP
    \item \textbf{Single point of attack}: Compromising the TTP compromises all exchanges
\end{itemize}

\subsection{Our Contribution}

We show that the impossibility can be circumvented by \emph{not having
anyone go first}. Instead of Alice sending her item, then Bob sending his,
we construct a protocol where the exchange item \emph{emerges} from
bilateral contribution.

The key insight: if neither party holds the exchange item until both
have contributed, there is nothing to cheat with.

\section{The Third Can of Paint}

\subsection{The Metaphor}

Consider Alice with red paint and Bob with blue paint. They want to
create purple paint together, but neither wants to give their paint first
(the trust problem).

\textbf{Traditional solution}: A trusted third party holds both cans,
mixes them, and distributes the purple paint.

\textbf{Our solution}: The purple paint is not ``held'' by anyone.
It \emph{emerges} when both colors are mixed. Neither party gives
anything to a holder---they contribute to a shared construction.

The ``third can'' (the purple paint) has special properties:
\begin{enumerate}
    \item It doesn't exist until both contribute
    \item Neither party holds it alone
    \item When it exists, both can access it
    \item Mathematics serves as the ``trusted'' mixer
\end{enumerate}

\subsection{Formal Construction}

We model the bilateral state with four boolean variables:
\begin{align*}
    d_a &: \text{Alice's contribution arrived} \\
    d_b &: \text{Bob's contribution arrived} \\
    a\_responds &: \text{Alice responded to the bilateral state} \\
    b\_responds &: \text{Bob responded to the bilateral state}
\end{align*}

The shared construct $V$ emerges only when both contributions exist:
\[
V\_emerges(d_a, d_b) =
\begin{cases}
    \text{Some}(v) & \text{if } d_a \land d_b \\
    \text{None} & \text{otherwise}
\end{cases}
\]

Each party's response requires $V$ to exist:
\begin{align*}
    alice\_response(V, d_b) &=
    \begin{cases}
        \text{Some}(r_a) & \text{if } V = \text{Some}(\_) \land d_b \\
        \text{None} & \text{otherwise}
    \end{cases} \\
    bob\_response(V, d_a) &=
    \begin{cases}
        \text{Some}(r_b) & \text{if } V = \text{Some}(\_) \land d_a \\
        \text{None} & \text{otherwise}
    \end{cases}
\end{align*}

The \textbf{third can} (shared artifact) emerges only when all three
components exist:
\[
third\_can(V, r_a, r_b) =
\begin{cases}
    \text{Some}(artifact) & \text{if all three are Some} \\
    \text{None} & \text{otherwise}
\end{cases}
\]

\section{Main Results}

\begin{theorem}[No Unilateral Completion]
Neither party alone can create the shared artifact:
\begin{align*}
    \forall d_b, a\_responds, b\_responds: &\quad
        make\_state(\text{false}, d_b, a\_responds, b\_responds) = \text{None} \\
    \forall d_a, a\_responds, b\_responds: &\quad
        make\_state(d_a, \text{false}, a\_responds, b\_responds) = \text{None}
\end{align*}
\end{theorem}

\begin{proof}
By construction, $V\_emerges$ requires $d_a \land d_b$. If either is
false, $V = \text{None}$, and therefore $third\_can = \text{None}$.
\end{proof}

\begin{theorem}[Atomicity]
The shared artifact exists if and only if all four conditions hold:
\[
(make\_state(d_a, d_b, a\_responds, b\_responds)).\text{isSome} \iff
d_a \land d_b \land a\_responds \land b\_responds
\]
\end{theorem}

\begin{proof}
Verified by exhaustive case analysis in Lean 4 (16 cases).
\end{proof}

\begin{theorem}[Symmetric Outcomes]
The outcome is always symmetric---both parties succeed or both fail:
\[
\forall d_a, d_b, a\_responds, b\_responds: \quad
outcome \in \{\text{BothSucceed}, \text{BothFail}\}
\]
\end{theorem}

\begin{proof}
The outcome type has only two constructors. By construction, there is
no ``AliceSucceeds\_BobFails'' or vice versa.
\end{proof}

\begin{theorem}[Fair Exchange Without TTP]
The bilateral construction satisfies the fair exchange specification:
\begin{enumerate}
    \item \textbf{Fairness}: If the artifact exists, both contributed
    \item \textbf{Atomicity}: Artifact exists iff full bilateral completion
    \item \textbf{No TTP}: Computation is purely deterministic from inputs
    \item \textbf{Termination}: Every input produces a definite outcome
\end{enumerate}
\end{theorem}

\begin{proof}
See \texttt{FairExchangeStandalone.lean} for the complete formal proof.
\end{proof}

\section{Relation to the Two Generals Problem}

This construction derives from the Two Generals Protocol (TGP)
\cite{tgp2026}, which solves the coordinated attack problem through
the same principle.

In TGP:
\begin{itemize}
    \item $d_a$: Alice's double-proof ($D_A$) delivered to Bob
    \item $d_b$: Bob's double-proof ($D_B$) delivered to Alice
    \item The ``attack key'' is the third can---it exists only when
          both generals have completed the bilateral construction
\end{itemize}

The generals don't ``decide'' to attack. The attack capability
\emph{emerges} from their collaboration. If either fails to contribute,
the capability doesn't exist, and both abort.

\section{Concrete Instantiation: Bilateral Schnorr Signatures}

\subsection{The Irrevocability Problem}

The abstract bilateral construction assumes the ``shared artifact'' doesn't
exist until both parties contribute. But in practice, when Alice computes
a partial Schnorr signature, that partial \emph{exists}---it's irrevocable.
If she sends it to Bob before knowing Bob will reciprocate, Bob could
complete the signature alone. How do you create a signature protocol where
neither party can compute their partial until bilateral completion?

\subsection{The Key Insight: Bilateral Challenge Binding}

In standard MuSig-style Schnorr multisignatures, each party computes:
\[
s_i = k_i + e \cdot a_i \cdot x_i
\]
where $k_i$ is the nonce, $e$ is the challenge, $a_i$ is the key coefficient,
and $x_i$ is the private key. The challenge $e$ is typically:
\[
e = H(R, P, m)
\]
where $R$ is the combined nonce point, $P$ is the aggregated public key,
and $m$ is the message.

\textbf{The problem}: Once Alice knows $R$ and $P$ (after nonce exchange),
she can compute $e$ and therefore her partial $s_A$. She could send $s_A$
before Bob sends $s_B$, creating asymmetry.

\textbf{Our solution}: Bind the challenge to the \emph{attack key}, which
can only be computed after bilateral T-proof exchange:
\[
e = H(R, P, m, \mathbf{attack\_key})
\]
where
\[
\mathbf{attack\_key} = H(T_A \| T_B)
\]

Since the attack key requires \emph{both} $T_A$ and $T_B$, and $T_B$ can
only exist if Bob completed the bilateral D-exchange, \textbf{neither party
can compute the challenge $e$ until bilateral completion}.

No challenge $\Rightarrow$ no partial signature $\Rightarrow$ nothing to
``go first'' with.

\subsection{The Four-Phase Protocol}

The bilateral Schnorr construction uses four phases:

\begin{center}
\begin{tabular}{llp{7cm}}
\toprule
\textbf{Phase} & \textbf{Artifacts} & \textbf{What It Achieves} \\
\midrule
C & $C_A, C_B$ & Unilateral commitments: public keys and nonce points \\
D & $D_A, D_B$ & Bilateral at C level: $D_X = \text{Sign}_X(C_X, C_Y)$ \\
T & $T_A, T_B$ & Bilateral at D level: $T_X = \text{Sign}_X(D_X, D_Y)$ --- \textbf{THE KNOT} \\
S & $S_A, S_B$ & Partial signatures with attack key binding \\
\bottomrule
\end{tabular}
\end{center}

\textbf{Phase C}: Each party floods their commitment containing their
public key $P_i$ and nonce point $R_i = k_i \cdot G$. No bilateral state yet.

\textbf{Phase D}: Upon receiving the counterparty's C, each party constructs
a signed double-proof embedding both commitments:
\[
D_A = \text{Sign}_A(C_A \| C_B) \qquad D_B = \text{Sign}_B(C_B \| C_A)
\]
D proves ``I saw your commitment.'' D embeds the counterparty's C---bilateral
at the C level.

\textbf{Phase T}: Upon receiving the counterparty's D, each party constructs
a signed triple-proof embedding both D proofs:
\[
T_A = \text{Sign}_A(D_A \| D_B) \qquad T_B = \text{Sign}_B(D_B \| D_A)
\]
T embeds both D proofs---bilateral at the D level. This is \textbf{the knot}.
Crucially: \emph{no partial signatures have been computed yet}.

\textbf{Phase S}: Upon receiving the counterparty's T, each party:
\begin{enumerate}
    \item Computes the attack key: $\text{attack\_key} = H(T_A \| T_B)$
    \item Computes the bilateral challenge: $e = H(R, P, m, \text{attack\_key})$
    \item Computes their partial signature: $s_i = k_i + e \cdot a_i \cdot x_i$
    \item Constructs $S_i$ containing their T, the counterparty's T, and the partial
\end{enumerate}

\textbf{Completion}: Upon receiving the counterparty's S, each party combines:
\[
s = s_A + s_B \qquad R = R_A + R_B
\]
The final signature $(R, s)$ verifies against the aggregated public key $P$.

\subsection{Why This Achieves Fair Exchange}

The construction guarantees bilateral determination:

\begin{theorem}[Bilateral Challenge Binding]
Neither party can compute the Schnorr challenge $e$ without possessing
both $T_A$ and $T_B$.
\end{theorem}

\begin{proof}
The challenge is $e = H(R, P, m, \text{attack\_key})$ where
$\text{attack\_key} = H(T_A \| T_B)$. Without both T proofs, the attack
key cannot be computed, and therefore $e$ cannot be computed.
\end{proof}

\begin{theorem}[No Unilateral Partial]
Neither party can compute their partial signature $s_i$ without the
counterparty having completed the T-phase.
\end{theorem}

\begin{proof}
$s_i = k_i + e \cdot a_i \cdot x_i$ requires knowing $e$. By the previous
theorem, $e$ requires both T proofs. $T_B$ can only exist if Bob had $D_A$,
which proves Bob completed the D-exchange. Therefore Alice cannot compute
$s_A$ unless Bob reached phase T.
\end{proof}

\begin{theorem}[Symmetric Outcomes]
Under any channel behavior (including total failure), the outcome is
symmetric: both parties obtain the signature, or neither does.
\end{theorem}

\begin{proof}
If $T_B$ never reaches Alice: Alice cannot compute attack\_key, cannot
compute $e$, cannot compute $s_A$, cannot attack.

If $T_A$ never reaches Bob: Bob cannot compute attack\_key, cannot
compute $e$, cannot compute $s_B$, cannot attack.

If $S_B$ never reaches Alice: Alice has attack\_key but not $s_B$,
cannot combine, cannot complete.

If $S_A$ never reaches Bob: Bob has attack\_key but not $s_A$,
cannot combine, cannot complete.

In all cases, failure is symmetric.
\end{proof}

\subsection{MuSig-Style Key Aggregation}

To prevent rogue-key attacks (where a malicious party chooses their public
key as a function of the honest party's key), we use MuSig-style key
coefficients:

\[
L = H(\text{``MUSIG\_KEYSET''} \| P_A \| P_B)
\]
\[
a_i = H(\text{``MUSIG\_COEF''} \| L \| P_i)
\]

The aggregated public key is:
\[
P = a_A \cdot P_A + a_B \cdot P_B
\]

Each party's partial signature uses their coefficient:
\[
s_i = k_i + e \cdot a_i \cdot x_i
\]

The combined signature verifies as:
\[
s \cdot G \stackrel{?}{=} R + e \cdot P
\]

\subsection{The Attack Key IS the Consent}

In TGP, the attack key answers: ``Can we attack?''

In fair exchange, the attack key answers: ``Did both parties consent?''

Same construction, different interpretation. The attack key is not just
\emph{evidence} of consent---it \emph{is} consent, crystallized into a
cryptographic binding that makes the signature possible.

The signature doesn't prove that consent happened. The signature
\emph{couldn't exist} without consent happening.

\subsection{Comparison with Prior Approaches}

\begin{center}
\begin{tabular}{lp{5cm}p{5cm}}
\toprule
\textbf{Approach} & \textbf{Mechanism} & \textbf{Limitation} \\
\midrule
2-of-2 threshold & Both parties must combine simultaneously & Requires synchronous finalization \\
Adaptor signatures & Partial becomes valid when secret revealed & One party knows secret first \\
HTLCs & Hash locks with timeouts & Free option problem \\
\textbf{Bilateral Schnorr} & Challenge bound to bilateral state & \textbf{Neither can compute partial first} \\
\bottomrule
\end{tabular}
\end{center}

\subsection{Implementation}

A reference implementation in Rust is available at
\texttt{github.com/riff-labs/thirdcan} under AGPLv3. The implementation
uses the secp256k1 curve (k256 crate) with SHA-256 for all hash functions.

Key implementation details:
\begin{itemize}
    \item All D, T, S proofs carry real Schnorr signatures that are verified
    \item Canonical ordering ensures both parties compute identical attack keys
    \item Transcript consistency checks prevent proof substitution attacks
    \item The emergence proof $H(S_A \| S_B)$ is embedded in the final signature
\end{itemize}

\section{Applications}

\subsection{Contract Signing}

Alice and Bob want to sign a contract. Neither wants to be bound while
the other is not.

Using the third can construction:
\begin{enumerate}
    \item Alice sends her commitment (proof she will sign)
    \item Bob sends his commitment (proof he will sign)
    \item The signed contract \emph{emerges} when both commitments exist
    \item Neither holds a signed contract until both do
\end{enumerate}

\subsection{Atomic Swaps}

Alice has BTC, Bob has ETH. They want to swap without a centralized
exchange.

Hash Time-Locked Contracts (HTLCs) attempt this but have a critical flaw:
\begin{enumerate}
    \item Alice creates secret $s$ and hash $H(s)$---\textbf{Alice knows $s$}
    \item Alice locks her BTC with hash lock $H(s)$
    \item Bob locks his ETH with the same hash lock
    \item Alice can reveal $s$ anytime before timeout to claim ETH
    \item Bob then uses $s$ to claim BTC
\end{enumerate}

The problem: Alice has \emph{optionality}. She knows $s$ from the start
and can wait, watch the market, and decide whether to complete. If prices
move against her, she lets the timelock expire. Bob's funds are locked
the entire time. This ``free option problem'' makes HTLCs only
\emph{approximately} fair---and has been exploited in practice through
MEV extraction and atomic swap griefing attacks \cite{daian2020}.

True bilateral construction requires neither party to hold the unlock
key until both have committed. In TGP terms: the attack key doesn't exist
until $d_a \land d_b$---neither party ``knows the secret'' because the
secret doesn't exist yet. A proper atomic swap would derive the unlock
key from \emph{both} parties' contributions, not from one party's
pre-existing secret.

\subsection{Escrow Without Escrow}

Traditional escrow requires a trusted holder. The third can construction
provides ``escrow'' without any holder:
\begin{itemize}
    \item No party holds the funds/items alone
    \item The ``escrow'' is the bilateral construction itself
    \item Mathematics serves as the impartial arbiter
\end{itemize}

\section{Discussion}

\subsection{What About the Impossibility Result?}

Pagnia and G\"artner's impossibility assumes that one party must
``go first''---that is, make an irrevocable commitment before the
other. Our construction avoids this by having neither party make
an irrevocable commitment. The commitments are \emph{conditional} on
the bilateral state, which doesn't exist until both contribute.

\subsection{Relation to Gradual Release}

Even, Goldreich, and Lempel \cite{even1980} proposed \emph{gradual release}
protocols where parties incrementally reveal their items bit by bit.
If one party stops early, both have approximately equal partial information.
This achieves ``approximate fairness'' through incremental commitment.

Our approach differs fundamentally: rather than gradually releasing a
pre-existing item, we construct an item that \emph{doesn't exist} until
both parties complete. There is no partial state---either the third can
exists (both succeed) or it doesn't (both fail). This yields exact fairness
rather than approximate fairness, at the cost of requiring the bilateral
construction to complete.

\subsection{The Role of the Channel}

Our construction assumes a \emph{fair-lossy} channel, defined precisely:

\begin{definition}[Fair-Lossy Channel]
A channel is \emph{fair-lossy} if for every message $m$ sent infinitely
often, there exists a finite time $t$ such that $m$ is delivered by time $t$.
Equivalently: persistent sending eventually succeeds. The channel may lose
any finite prefix of transmissions, but cannot lose all transmissions of
a message sent infinitely often.
\end{definition}

This is strictly weaker than reliable delivery (which guarantees every
send succeeds) but strictly stronger than fully adversarial (which permits
permanent partition). Fair-lossy captures realistic networks: packets drop,
but persistent retransmission works.

Under fair-lossy, the bilateral construction completes: both parties
flood continuously, so both contributions eventually arrive.

Under a fully adversarial channel (permanent partition), both parties
timeout and abort---a symmetric outcome. The construction degrades
gracefully: no asymmetric failures are possible regardless of channel
behavior.

\subsection{Mathematics as TTP}

In our construction, mathematics serves the role traditionally played
by a TTP:
\begin{itemize}
    \item \textbf{Deterministic}: Same inputs always produce same outputs
    \item \textbf{Incorruptible}: Cannot be bribed or compromised
    \item \textbf{Available}: Always ``online'' (it's just computation)
    \item \textbf{Verifiable}: Anyone can check the construction
\end{itemize}

\section{Conclusion}

We have shown that fair exchange without a trusted third party is
possible through bilateral construction. The key insight is that
neither party ``goes first''---the exchange item emerges from their
collaboration.

The construction is simple, formally verified, and applicable to a
wide range of problems including contract signing, atomic swaps, and
escrow services.

The ``third can of paint'' is not held by Alice, not held by Bob, and
not held by any third party. It emerges from mathematics itself.

\begin{thebibliography}{9}

\bibitem{pagnia1999}
H. Pagnia and F. C. G\"artner,
``On the Impossibility of Fair Exchange without a Trusted Third Party,''
Technical Report TUD-BS-1999-02, Darmstadt University of Technology, 1999.

\bibitem{tgp2026}
Wings,
``The Two Generals Protocol: Deterministic Coordination Over Lossy Channels,''
Riff Labs Technical Report, 2026.

\bibitem{asokan1998}
N. Asokan, V. Shoup, and M. Waidner,
``Optimistic Fair Exchange of Digital Signatures,''
EUROCRYPT 1998.

\bibitem{even1980}
S. Even, O. Goldreich, and A. Lempel,
``A Randomized Protocol for Signing Contracts,''
Communications of the ACM, vol.~28, no.~6, pp.~637--647, 1985.

\bibitem{daian2020}
P. Daian, S. Goldfeder, T. Kell, Y. Li, X. Zhao, I. Bentov, L. Breidenbach, and A. Juels,
``Flash Boys 2.0: Frontrunning in Decentralized Exchanges, Miner Extractable Value, and Consensus Instability,''
IEEE Symposium on Security and Privacy, 2020.

\bibitem{frost2020}
C. Komlo and I. Goldberg,
``FROST: Flexible Round-Optimized Schnorr Threshold Signatures,''
Selected Areas in Cryptography (SAC), 2020.

\end{thebibliography}

\end{document}
